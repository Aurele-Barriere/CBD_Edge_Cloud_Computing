\documentclass[utf8,xcolor=table, page number]{earlywinter}

\usepackage[T1]{fontenc}
\usepackage[frenchb]{babel}

\begin{document}
\title{Edge Cloud Computing}
\author{Aurèle BARRIÈRE & Solène MIRLIAZ}

\begin{frame}[plain]
  \titlepage%
\end{frame}

% ----------------
% - Introduction -
% ----------------
\section{Introduction}
\begin{frame}
	\frametitle{Introduction}

\end{frame}

  \begin{frame}
    \frametitle{Table of contents}
    \tableofcontents[]
  \end{frame}
  
  
\section{Edge computing challenges}
\subsection{Internet of Things and Limits of the clouds}
\begin{frame}
  \frametitle{Internet of Things and Limits of the clouds}
  Evolution of the number of connected objects
  
	\begin{alertblock}{Problem}
			The network can't handle the workload and the communication.
	\end{alertblock}
        \begin{alertblock}{Problem}
			The Cloud often introduces too much latency.
	\end{alertblock}

\end{frame}

\subsection{Bringing the cloud to the Edge: an hybrid system} %I'm not sure this is a good phrasing. maybe some of the papers we read would disagree
\begin{frame}
  \frametitle{Edge computing}
  \framesubtitle{Principle}
  Computing on the end-node of the network (the "things")\\
  -> \emph{Edge} of the network
\end{frame}

\begin{frame}
  \frametitle{Edge-Cloud computing}
  \framesubtitle{An hybrid system}
\end{frame}

\subsection{Challenges}

% the challenges and research interests are almost the same to me

\begin{frame}
  \frametitle{Challenges}
  \framesubtitle{Energy}
  
  Mobile device with low battery
  
\end{frame}
\begin{frame}
  \frametitle{Challenges}
  \framesubtitle{Computing power}
  
  Increasing
  
\end{frame}
\begin{frame}
  \frametitle{Challenges}
  \framesubtitle{Mobility}
  
  Mobile things. Failure-tolerence must be high.
  
  Where to put the service in the edge?
  
\end{frame}
\begin{frame}
  \frametitle{Challenges}
  \framesubtitle{Heterogenity}
  
  Of the network and things' abilities.
  
	\begin{example}
		Cellphone, Oven, radio, heater, computer, etc.
	\end{example}
  
  Of the programming languages used by each.  
  
  Of the applications
	\begin{example}
		High CPU need, high memory need, quick communication, etc.
	\end{example}
  
\end{frame}

\section{Current research studies}
\subsection{Architecture}
\begin{frame}
  \frametitle{Current Research}
  \framesubtitle{Architecture}
\end{frame}
\subsection{Workload Distribution}
\begin{frame}
  \frametitle{Current Research}
  \framesubtitle{Workload Distribution}
\end{frame}
\subsection{Problem Identification and Heuristics}
\begin{frame}
  \frametitle{Current Research}
  \framesubtitle{Problem Identification and Heuristics}
\end{frame}
\subsection{Programmability}
\begin{frame}
  \frametitle{Current Research}
  \framesubtitle{Programmability}
\end{frame}
\subsection{Naming}
\begin{frame}
  \frametitle{Current Research}
  \framesubtitle{Naming}
\end{frame}

\section{Case example: a traffic light}
\begin{frame}
  \frametitle{Case Example}
  \framesubtitle{Smart Traffic Light}
\end{frame}
\section{Conclusion}
\begin{frame}
  \frametitle{Conclusion}
  Hybrid system needed.
  No cure-all solution.
  
  What about the economical aspect ?
  
\end{frame}
\end{document}
